%\myChapter{Titre}{Titre court}
\myChapter{Généralités sur les micro-centrales hydroélectrique}{site d'exploitation}
\label{chapEtatDeLart}
%\myMinitoc{Profondeur de la minitoc (section|subsection|subsubsection)}{Titre de la minitoc}
\myMiniToc{section}{Contents}

Le travail que nous menons dans ce mémoire s'inscrit dans le domaine de recherche du CSCW (\textit{Computer Supported Cooperative Work}) ou TCAO (\textit{Travail Coopératif Assisté par Ordinateur}) en français. Nous considérons un système à flots de tâches (worflow system) dont les acteurs, géographiquement distants, coordonnent leurs activités par échange de documents électroniques qu'ils éditent de façon désynchronisée. Les systèmes logiciels chargés d'assurer la coopération entre les différents acteurs doivent présenter certaines caractéristiques afin de faire face aux contraintes du CSCW. Dans ce chapitre nous présentons la notion de travail coopératif (sect. \ref{sectionTravailCooperatif}) en faisant le parallèle avec le CSCW (sect. \ref{sectionCSCW}) et les systèmes de CSCW...


%SECTION TRAVAIL COOPERATIF
%\mySection{Titre}{Titre court}
\mySection{Le travail coopératif}{}\label{sectionTravailCooperatif}
Le terme...
\begin{proof}
La preuve
\end{proof}

\mySubSection{L'organisation du travail}{}\label{sectionOrganisationTravail}
Sur le plan du travail...

\begin{example}\textit{Le cultivateur et son champ}\\
Un cultivateur...
\end{example}

\mySubSection{Notion de Travail Coopératif Assisté par Ordinateur (CSCW)}{}\label{sectionCSCW}
La collaboration...

\mySubSubSection{Les caractéristiques des systèmes de CSCW}{}\label{sectionCaractSystCSCW}
La mise...

\myDescription{Système distribué}{Dans un contexte distribué, chaque site possède des copies locales (répliques) des objets partagés et c'est donc sur ces copies que sont portées les contributions locales. Pour obtenir un état global, le système synchronise toutes les répliques. Par conséquent, il est crucial de mettre en place une procédure de contrôle de la concurrence et ce pour assurer la convergence des copies vers un même état.}

\myDescription{Système non distribué}{Dans...}

\myDescription{Système synchrone}{Le CSCW est synchrone lorsque les mises à jour apportées par un acteur sur les données partagées sont immédiatement (en un intervalle de temps raisonnable) visibles par l'ensemble des acteurs pouvant avoir accès à ces données. Ces systèmes sont dits temps réel. Les éditeurs collaboratifs temps réel (ou éditeurs WYSIWIS\footnote{What You See Is What I See pouvant être traduit en \textit{ce que vous voyez est ce que je vois}.}) tels que \textit{Etherpad}\footnote{Etherpad est un éditeur collaboratif temps réel disponible à l'adresse \url{http://www.etherpad.org/}.} et \textit{Google Docs}\footnote{L'une des fonctionnalités de Google Docs est la possibilité de réaliser de l'édition temps réel et à plusieurs, \url{https://docs.google.com/}.} en sont de parfaites illustrations.}

\myDescription{Système asynchrone}{Le...}

\mySubSubSection{Classification des systèmes de CSCW}{Classification des systèmes de CSCW}
L'analyse...
\begin{figure}[h!]
    \centering
		\includegraphics[scale=0.7]{chap1/images/categorizationJoh.png}
    \caption{Matrice $2\times 2$ de \textit{Johansen} pour la catégorisation des systèmes de CSCW.}
	 \label{figCscwToolCateg1}
\end{figure}



\mySection{Synthèse}{}\label{sectionSyntheseChap1}
Dans...

